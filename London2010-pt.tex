% Setup appearance:
\documentclass{beamer}
\usetheme{Darmstadt}
\usefonttheme[onlylarge]{structurebold}
\setbeamerfont*{frametitle}{size=\normalsize,series=\bfseries}
\setbeamertemplate{navigation symbols}{}


% Standard packages

\usepackage[english]{babel}
\usepackage[latin1]{inputenc}
\usepackage{times}
\usepackage[T1]{fontenc}


% Setup TikZ
\usepackage{tikz}
\usetikzlibrary{arrows}
\tikzstyle{block}=[draw opacity=0.7,line width=1.4cm]


% Author, Title, etc.

\title[Robotic Validation of AFM and Beyond] 
{%
  Robotic Validation of AFM, Scale-freeness, Local Communication etc.%
}

\author[MOFSarker]
{
  Md Omar Faruque Sarker
}

\institute[UWN]
{
 Robotic Intelligence Lab\\
 University of Wales, Newport
}

\date{May 2010}



% The main document

\begin{document}

\begin{frame}
  \titlepage
\end{frame}

\begin{frame}{Outline}
  \tableofcontents
\end{frame}


\section{Introduction}

\subsection{The Model and the Problem}

\begin{frame}{What is haplotyping and why is it important?}
  You hopefully know this after the previous three talks\dots
\end{frame}

\begin{frame}[t]{General formalization of haplotyping.}
  \begin{block}{Inputs}
    \begin{itemize}
    \item A \alert{genotype matrix} $G$.
    \item The \alert{rows} of the matrix are \alert{taxa / individuals}.
    \item The \alert{columns} of the matrix are \alert{SNP sites /
        characters}. 
    \end{itemize}
  \end{block}
  \begin{block}{Outputs}
    \begin{itemize}
    \item A \alert{haplotype matrix} $H$.
    \item Pairs of rows in $H$ \alert{explain} the rows of $G$.
    \item The haplotypes in $H$ are \alert{biologically plausible}. 
    \end{itemize}
  \end{block}
\end{frame}


\begin{frame}[t]{Our formalization of haplotyping.}
  \begin{block}{Inputs}
    \begin{itemize}
    \item A genotype matrix $G$.
    \item The rows of the matrix are individuals / taxa.
    \item The columns of the matrix are SNP sites / characters.
    \item<alert@1->
      The problem is directed: one haplotype is known.
    \item<alert@1->
      The input is biallelic: there are only two homozygous
      states (0 and 1) and one heterozygous state (2).
    \end{itemize}
  \end{block}
  \begin{block}{Outputs}
    \begin{itemize}
    \item A haplotype matrix $H$.
    \item Pairs of rows in $H$ explain the rows of $G$.
    \item<alert@1> The haplotypes in $H$ form a perfect phylogeny.
    \end{itemize}
  \end{block}
\end{frame}


\end{document}


