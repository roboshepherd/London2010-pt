% Setup appearance:
%\documentclass[handout,draft]{beamer}
\documentclass{beamer}
\usetheme{Frankfurt} % right index Marburg
\usefonttheme[onlylarge]{structurebold}
\setbeamerfont*{frametitle}{size=\normalsize,series=\bfseries}
\setbeamertemplate{navigation symbols}{}
% Standard packages
\usepackage[english]{babel}
\usepackage[latin1]{inputenc}
\usepackage{times}
\usepackage[T1]{fontenc}
\usepackage{makeidx} % allows for indexgeneration
\usepackage{graphicx}
\usepackage{multicol}
\usepackage{subfigure}
\usepackage{mathptmx} % use Times fonts if available on your TeX system
\usepackage{setspace}
\usepackage{epsfig}
\usepackage{array}
% Setup TikZ
\usepackage{tikz}
\usetikzlibrary{arrows}
\tikzstyle{block}=[draw opacity=0.7,line width=1.4cm]

% Author, Title, etc.
\title[Self-organized semantic NeLI]
{%
  Self-organized Semantic NeLI
  } 
\subtitle
{
Through Self-organized Approaches Using Semantic Web, Social Networking and  Web 2.0 Frameworks %Using Semantic Web, Web 2.0 and Social Networking Tools %
}

\author[MOFSarker]
{
  M. O. Faruque Sarker
}

\institute[UWN]
{
 PhD Candidate\\
 Cognitive Robotics Research Centre\\
 Newport Business School\\
 University of Wales, Newport
}

\date{2 November 2010}



% The main document

\begin{document}

\begin{frame}
  \titlepage
\end{frame}

\begin{frame}{Outline}
  \tableofcontents
\end{frame}
%%%===========================================================
\section{Introduction}
\begin{frame}[t]{Background: Web 2.0, Semantic web and Self-organization}
\vspace*{-0.15cm}
\begin{block}{Why self-organization is relevant?}
\begin{itemize}
\item \small Web 2.0 is based on \alert{bottom-up processes}\\ 
$\rightarrow$ \scriptsize users generate contents through explicit and implicit social interactions.
\item \small Complex patterns can be resulted from \alert{simple behavioural rules}\\ 
$\rightarrow$ \scriptsize complexity is not necessarily outcome of sophisticated agent cognition
\end{itemize}
\end{block}
%%
\begin{columns}
\column{0.7\textwidth}
\vspace*{-0.15cm}
\begin{block}{A Self-organized approach}
\begin{itemize}
\item \small \alert{Pattern formation} through the local interactions internal to the system \footnote{\tiny Camazine \textit{et al.}, Self-organization in Biological Systems, 2001.}
%\item \small Shows \alert{four perspectives:\footnote{\tiny Camazine \textit{et al.}, Self-organization in Biological Systems, 2001.}}\\
%\scriptsize i) positive feedback, ii) negative feedback, iii) multiple interactions and iv) randomness
\item \small Needs \alert{four elements:}\footnote{\tiny Sarker \& Dahl. \textit{LNCS} 6234, 2010.}\\
\scriptsize i) continuous flow of information, ii) concurrence, iii) learning and iv) forgetting of agents
\item \small \alert{Scalable, adaptable and robust} solutions\\
$\rightarrow$ \scriptsize Depends on less communication/computation, almost no user/environment modelling
\end{itemize}
\end{block}
%%
\hspace*{-0.25cm}
\column{0.3\textwidth}
\vspace*{-0.4cm}
\begin{figure}
\centering
\includegraphics[width=0.99\textwidth, angle=0]
{/media/Preload/Pub2010/ThoughtsLinedUp/photos/termites_nest.eps}
%figure caption is below the figur
\caption{\scriptsize A termite nest from bottom-up approach}
\label{fig:afm} % Give a unique label
\end{figure}
%%
\end{columns}
\end{frame}
%%=====================================================
\section{A Semantic Search and Navigation System}
\begin{frame}[t]{Self-organized Semantic Search through Attractive Field Model}
\vspace*{-0.25cm}
  \begin{columns}
\column{0.5\textwidth}
\vspace*{-0.5cm}
\begin{figure}
\centering
\includegraphics[height=0.65\textwidth, angle=0]
{/media/Preload/Pub2010/ThoughtsLinedUp/dia-files/AFM-Diag-abstract2.eps}
\vspace*{-0.5cm}
\caption{\scriptsize The attractive filed model (AFM)}
\label{fig:afm} % Give a unique label
\end{figure}
\column{0.55\textwidth}
\vspace*{-0.25cm}
\begin{scriptsize}
      \begin{tabular}{m{0.85in}|m{1.3in}}
      \hline
      Source nodes (o) & \scriptsize \alert{items} to be searched\\      \hline
      Agent nodes (x) & \scriptsize \alert{user search agents} e.g.,  mozilla browser\\
     \hline
     Black solid edges & \scriptsize \alert{attractive fields} that correspond to an agent's stimuli for \alert{related items}\\
	\hline
	Green edges &  \alert{attractive fields of unrelated items}  shown as  (w)\\
	%\hline
	%Black dashed edges & not edges, but shows an agent found an item.\\% at any point in time\\
	\hline
      \end{tabular}
\end {scriptsize}
\vspace*{-0.25cm}
\end{columns}
%%
\small \alert{Strength of an attractive field:} 
\alert{$ S_{j}^{i} \propto \frac{k_{j}^{i}}{d_{ij}} \phi _{j}\ $}
%\vspace*{-0.25cm}
%\begin{equation}
%\[
%\scriptsize
%S_{j}^{i} \propto \frac{k_{j}^{i}}{d_{ij}+\delta } \phi _{j}\
%\]
%%\label{eqn:afm1}
%\end{equation}
%\vspace*{-0.25cm}
\begin{columns}
\column{0.5\textwidth}
\vspace*{-0.25cm}
\begin{block}{Parameters}

\small
\alert{$k_{j}^{i}$}   $i$ agent's \alert{sensitization} for item  $j$\\
$\rightarrow$ \scriptsize How much $i$ is sensitive to $j$ item?\\
\small
\alert{$d_{ij}$:} $i$ agent's \alert{location preference}\\
$\rightarrow$ \scriptsize How far $j$ is located from $i$?\\
\small
\alert{$\phi _{j}$:} \alert{relative urgencies} of item $j$.\\
$\rightarrow$ \scriptsize What is the level of urgency of $j$?  
\end{block}

\column{0.5\textwidth}
\vspace*{-0.45cm}
\begin{block}{Learning search patterns}
%\scriptsize
\small E.g.: Increasing item sensitization of agents by a rate $k_{INC}$\\
\vspace*{0.25cm}
%%With  an agent's \textit{rate of learning} items, $k_{INC}$:
%\begin{equation}
\alert{ 
$ If\hspace*{0.15cm}item\hspace*{0.15cm}is\hspace*{0.15cm}rated:\hspace*{0.15cm}  k^i_j \rightarrow   k^i_j \hspace*{0.15cm} + \hspace*{0.15cm} k_{INC}
$ }\\
\vspace*{0.25cm}
$\rightarrow$ \scriptsize A form of \alert{positive feedback} 
%\end{equation} 
%\begin{equation}
%\alert{
% If\hspace*{0.15cm}item\hspace*{0.15cm}is\hspace*{0.15cm}not\hspace*{0.15cm}rated:\hspace*{0.15cm}  k^i_j \rightarrow   k^i_j \hspace*{0.15cm} - \hspace*{0.15cm} k_{DEC}
% }
%\label{eqn:k-dec}
%\end{equation}   	
\end{block}
\end{columns}
\end{frame}
%------------------------------------------------------------
\begin{frame}[t]{A Frameworks for Semantic Search and Navigation}
\vspace*{-0.25cm}
\begin{block}{Key aspects}
\begin{itemize}
\item \scriptsize \alert{Easy and interactive}, compatible, support for different users' \alert{personalization} \item \scriptsize \alert{Reasoning} new information from existing information
\end{itemize}
\end{block}
\begin{columns}
\column{0.6\textwidth}
\begin{figure}
\centering
\includegraphics[height=0.73\textwidth, angle=0]
{/media/Preload/Pub2010/ThoughtsLinedUp/dia-files/sematic-search-nav.eps}
%figure caption is below the figur
\caption{\scriptsize Python based semantic wiki framework}
\label{fig:ssnf} % Give a unique label
\end{figure}
\column{0.5\textwidth} 
%\scriptsize \alert{A framework for semantic web wiki adopted from open source semantic tools}\footnote{\scriptsize http://semanticweb.org/}
\vspace*{0.1cm}
\begin{scriptsize}
      \begin{tabular}{m{0.25\textwidth}|m{0.6\textwidth}}
      \hline
      \textbf{Model} & \textbf{Tools/Frameworks}\\
      \hline
      \alert{Database storage} &
      Domino, Plomino.dominoimport\\
      \hline      
	  \alert{Webpage /Knowledge base}	      
      & 
      XML, Resource \protect\newline Description Framework, RDF/XML\\	      
      \hline
      \alert{Data \protect\newline rendering} & XML, Dublin Core, SPARQL Protocol and RDF Query Language\\
      \hline
      \alert{Page transformation} & XML Stylesheet XSLT\\
  		  \hline
  	 \alert{Web \protect\newline publishing} & HTML/XHTML, JavaScript Object Notation, Wiki Exchange Format\\
  	\hline
  	 \alert{Backend} & Python web frameworks\\
  		  \hline
	  	 \alert{Frontend} & JavaScript/AJAX\\
  		  \hline  
\end{tabular}
\end{scriptsize}
\end{columns}  
\end{frame}
%%%===========================================================
\section{Web 2.0 Features for NeLI}
%%--------------------------------------------------
\begin{frame}[t]{Web 2.0 in NeLI: Wiki, Blogs, RSS, Podcast and more}
\vspace*{-0.25cm}
%\begin{block}{Key aspects}
%\begin{itemize}
%\item \scriptsize Massive \alert{user-generated} contents, up-to-date, \alert{open} and interoperable 
%\item \scriptsize \alert{Social networking} through communication and interactions
%\end{itemize}
%\end{block}

\begin{columns}
\column{0.6\textwidth}  	
\begin{block}{Adopting web 2.0 in NeLI}
\begin{itemize}
\item \alert{Blogs:} \small One-to-many interactive timely discussions %\\
$\rightarrow$ \scriptsize e.g. Dimov's \textit{Clinical Cases \& Images\footnote{\scriptsize http://casesblog.blogspot.com/}}
\item \small \alert{Wikis:} Bottom-up content creation, editing and discussions %\\
$\rightarrow$ \scriptsize e.g. Flu Wiki\footnote{\scriptsize http://www.fluwikie.com/}
\item \small \alert{RSS Feeds:} Up-to-date feeds on mobile devices \textit{anytime anywhere}
\item \small \alert{Social Bookmarking:} Collaborative tagging, link sharing %\\
$\rightarrow$ \scriptsize e.g. IBM's Dogear%\footnote{\scriptsize http://www.ibm.com/software/lotus/products/connections/dogear.html}
\item \small \alert{Podcast or Vodcast:} Audio-visual material for increasing awareness
\end{itemize}
\end{block}
%
\column{0.4\textwidth}  
%\vspace*{-0.25cm}	
\begin{block}{Constraints}
\begin{itemize}
\begin{small}
\item  Lack of \alert{authoritative control} over content \item  Lack of \alert{accuracy} 
\item  Information \alert{overload}
\item  \alert{Anonymity} %or lack of authorship
%\item  Requires \alert{monitoring}
\item Requires \alert{moderating}\\ 
$\rightarrow$  \scriptsize e.g. Ganfyd\footnote{\scriptsize  http://www.ganfyd.org/}
\item  \small \alert{Roll-back} is costly\\
\item  Requires enforcing policies for ensuring \alert{privacy and copyright} issues
\end{small}
\end{itemize}
\end{block}
\end{columns}  	
\end{frame}
%%%===========================================================
\section{Data-mining over Social Networks}
%%--------------------------------------------------
\begin{frame}[t]{A System for Exploiting Social Networks (SN)}	
%\begin{columns}
%\column{0.6\textwidth}  
\begin{block}{Mining SN}
\begin{itemize}
\item  \small Relevant page \alert{identification, pre-processing and extraction}\\
$\rightarrow$  \scriptsize Facebook/Twitter API
\item  \small \alert{Integration}\\ 
$\rightarrow$  \scriptsize  Probabilistic approach for name disambiguation, classification
\item  \small Database \alert{storage,  indexing and access}\\
$\rightarrow$  \scriptsize Domino, Python wrapper
\item \small Social network \alert{modelling}\\ 
$\rightarrow$  \scriptsize semantic modelling, clustering using statistical and socio-biological models
\item \small \alert{Search} services\\
$\rightarrow$ \scriptsize Hot topic, disease outbreak, expert search 
\end{itemize}
\end{block}
  	
\begin{block}{Disseminating public health information over SN}
\begin{itemize}
\item \small \alert{Automated delivery} based on SN clusters/ subscription
\item \small \alert{Relevant} information (recall self-organized search)
\item \small \alert{Low overhead} of information based on user feedback
\end{itemize}
\end{block}
  	
\end{frame}

%%%===========================================================
\section{Infectious Disease Monitoring through  Social Networks}
%%--------------------------------------------------
\begin{frame}[t]{A Framework for Infectious Disease Monitoring through SN}
\begin{columns}
\column{0.6\textwidth} 
\vspace*{-0.35cm} 	
\begin{block}{HealthMap:  Global Infectious Disease Monitoring Site}
\begin{itemize}
\item \small Popular \alert{news-based}\\
$\rightarrow$ \scriptsize alerts from  multi-sources, last 30 day stats%, alert by country %, Latest alerts
\item \small \alert{web 1.0 framework}\\ 
$\rightarrow$ \scriptsize no metadata storage, no knowledge-base, needs human  intervention
\end{itemize}
\end{block}
\vspace*{-0.2cm}   	
\begin{block}{Going Beyond HealthMap}
\begin{itemize}
\item \small \alert{Social-network based:}\\
$\rightarrow$ \scriptsize biological \& statistical models
\item \small \alert{Semantic web technology:}\\
$\rightarrow$ \scriptsize inferred knowledge 
\item \small \alert{Personalized} alerts\\ 
$\rightarrow$ \scriptsize over Facebook, Twitter, mobile devices
\item \small \alert{Self-organized} design and operation \\ 
$\rightarrow$ \scriptsize scalable, minimum site maintenance overhead, fault-tolerant and mostly automated
\end{itemize}
\end{block}

\column{0.5\textwidth}
\vspace*{-0.25cm} 	  
\begin{figure}
\centering
\includegraphics[width=0.9\textwidth, angle=0]
{HealthMap_process.eps}
\newline
\includegraphics[width=0.9\textwidth, angle=0]
{HealthMap_arch.eps}
%figure caption is below the figur
\caption{\scriptsize HealthMap architecture \protect\newline (Freifeld et al. \textit{J. of the American Medical Informatics Association}:15, 150, 2008)}
\label{fig:ssnf} % Give a unique label
\end{figure}

\end{columns}  	
\end{frame}

%%%===========================================================
\section{Conclusion and Outlook}
\begin{frame}[t]{Conclusion and Outlook}
  \begin{itemize}
    \item \small Web 2.0 is based on \alert{bottom-up processes}\\
    \item \small \alert{Self-organized approaches} has potential to manage NeLI's large semantic web system\\
    \item \small Both \alert{biological and statistical approaches} can be used to exploit the social networking tools for data mining and information dissemination\\
    \item \small \alert{NeLI's infectious disease monitoring framework} can potentially be built by putting semantic web technology into existing frameworks.\\
    \end{itemize}
\end{frame}

%%%%%%%%%%%%%%
\end{document}


