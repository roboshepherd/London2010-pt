% Setup appearance:
\documentclass{beamer}
\usetheme{Frankfurt} % right index Marburg
\usefonttheme[onlylarge]{structurebold}
\setbeamerfont*{frametitle}{size=\normalsize,series=\bfseries}
\setbeamertemplate{navigation symbols}{}
% Standard packages
\usepackage[english]{babel}
\usepackage[latin1]{inputenc}
\usepackage{times}
\usepackage[T1]{fontenc}
\usepackage{makeidx} % allows for indexgeneration
\usepackage{graphicx}
\usepackage{multicol}
\usepackage{subfigure}
\usepackage{mathptmx} % use Times fonts if available on your TeX system
\usepackage{setspace}
\usepackage{epsfig}
% Setup TikZ
\usepackage{tikz}
\usetikzlibrary{arrows}
\tikzstyle{block}=[draw opacity=0.7,line width=1.4cm]

% Author, Title, etc.
\title[Self-organized Multi-robot Task~Allocation] 
{%
  Self-organized Multi-robot Task~Allocation%
}

\author[MOFSarker]
{
  Md Omar Faruque Sarker
}

\institute[UWN]
{
 Cognitive Robotics Research Centre\\
 University of Wales, Newport
}

\date{October 2010}



% The main document

\begin{document}

\begin{frame}
  \titlepage
\end{frame}

\begin{frame}{Outline}
  \tableofcontents
\end{frame}
%%%===========================================================
\section{Introduction}
%%--------------------------------------------------
\begin{frame}[t]{Multi-robot Task Allocation (MRTA)}
	
\begin{block}{What is MRTA?}
In a multi-tasking environment dynamically allocate appropriate tasks to appropriate robots considering the  changes in \alert{task-requirements, team-performance and environment.}
\end{block}
  	
\begin{block}{Why MRTA is difficult?}
In typical large distributed multi-robot teams:
\begin{itemize}
\item No centralized planner or coordinator
\item \alert{Robots have limited ability}\\
$\rightarrow$ \small to sense, communicate and interact locally
\item \normalsize \alert{Robots have limited world-views}\\ 
$\rightarrow$ \small knowledge of past, present and future actions of others
\end{itemize}
\end{block}
  	
\end{frame}
%%--------------------------------------------------
\begin{frame}[t]{Major Approaches for MRTA}
 \begin{columns}
      \column{.5\textwidth}	
\begin{block}{Explicit allocation}
Through \alert{explicit modelling} of environment, tasks, robot capabilities. Some forms are: knowledge based, market based, role/value based, control theoretic. 
\begin{itemize}
\item \small Pros: Straight-forward to design, implement and analyse formally. % predictable
\item \small Cons: \alert{Not suitable for large teams ($>$ 10) and heavy dependency on explicit global broadcast communication.}
\end{itemize}
\end{block}
 \column{.5\textwidth}
\begin{block}{Self-organized allocation}
Through \alert{emergent group behaviour} produced by the local interaction and implicit or local communication. Most common form is: response threshold based approach.
\begin{itemize}
\item \small Pros: Suitable for large teams, no explicit model, implicit/local communication
\item \small Cons: \alert{Difficult to design, implement, analyse and limited to one specific global task.} % unpredictable
\end{itemize}
\end{block}
\end{columns}
\end{frame}
%%--------------------------------------------------
\begin{frame}[t]{Self-organization}
\begin{block}{What is Self-organization?}
Pattern formation in both biological and physical systems through the interactions internal to the system (Camazine et al. 2001).%\\
%$\rightarrow$ \small global patterns from local interactions.
\end{block}
\begin{block}{Why Self-organized approach to MRTA?}
\begin{itemize}
\item \normalsize \alert{Implementing simple agent behaviours is economical}\\
$\rightarrow$ \small no sophisticated cognitive agents.
\item \normalsize \alert{Easily scalable for large robot-teams and tasks}\\  
$\rightarrow$ \small no explicit modelling of environment.
\item \normalsize \alert{Fault-tolerant}\\ 
$\rightarrow$ \small no leaders, templates or blue-prints.
\item \normalsize \alert{Energy-efficient}\\
$\rightarrow$ \small no costly communication or computation overhead.
\end{itemize}
\end{block}
\end{frame}
%%--------------------------------------------------
\begin{frame}[t]{Ingredients of Self-organization}
\begin{columns}
\column{0.7\textwidth}
\begin{figure}
\centering
\includegraphics[height=0.7\textwidth, angle=0]
{/media/Preload/Pub2010/ThoughtsLinedUp/images/dia-files/self-org-1}
%figure caption is below the figur
\caption{\scriptsize The four ingredients of self-organization}
\label{fig:self-org} % Give a unique label
\end{figure}
\column{0.3\textwidth}
\scriptsize Examples:
      \begin{tabular}{l|p{1in}}
      \hline
      A & \scriptsize \alert{Ants' recruitments to food source} through trail laying/following\\
      \hline
     B & \scriptsize \alert{Overcrowding} at food sources\\
     \hline
     C & \scriptsize \alert{Various types of communications} through peer-to-peer, broadcast or stigmergic\\
     \hline
     D & \scriptsize \alert{Randomness} and/or error in trail-following that leads to discover new food sources\\
     \hline       
      \end{tabular}

\end{columns}
\end{frame}
%--------------------------------------------
\begin{frame}[t]{Self-regulation of an Agent}

\begin{columns}
\column{0.5\textwidth}
\begin{figure}
\centering
\includegraphics[height=0.5\textwidth, angle=0]
{/media/Preload/Pub2010/ThoughtsLinedUp/images/dia-files/self-org-agent}
%figure caption is below the figur
\caption{\scriptsize Three major interfaces of a self-organized agent}
\label{fig:self-org-agent} % Give a unique label
\end{figure}
\column{0.5\textwidth}
\scriptsize \alert{Self-organization in birds nesting}
      \begin{tabular}{p{0.7in}|p{1.2in}}
      \hline
      Simple \protect\newline behavioural rules & \scriptsize \alert{Follow: \textit{``I nest close where you nest \protect\newline ... \protect\newline unless overcrowded''}} \\
      \hline
     Local \protect\newline communication & \scriptsize \alert{Communications through local broadcast signals}\\
     \hline
     Local \protect\newline interactions & \scriptsize \alert{Courtship display with neighbours}\\
	\hline
      \end{tabular}

\end{columns}
\end{frame}
%%--------------------------------------------------
%%%===========================================================
\section{Task Allocation by Attractive Field Model (AFM)}
%%--------------------------------------------------
\begin{frame}[t]{Attractive Field Model (AFM)}	
\begin{block}{Features of AFM}
\begin{itemize}
\item \alert{Interdisciplinary:} \small From the observation of ant, human and robotic social systems. 
\item \alert{Abstraction:} \small Sufficient abstraction to accommodate different sensing and communication models.
\end{itemize}
\end{block}
  	
\begin{block}{Requirements of AFM}
\begin{enumerate}
\item \normalsize  \alert{Concurrence:} \small The simultaneous presence of several options or tasks, at least a single task and the option of not doing any task. 
\item \normalsize  \alert{Continuous flow of information:} \small Establish a flow of information to perceive tasks and receive feedback on system performance.
\item \normalsize  \alert{Sensitization:} \small Each individual must have different levels of preference or {\em sensitivity} to the available tasks.
\item \normalsize  \alert{Forgetting:} \small A mechanism by which the sensitisation levels are reduced or {\em forgotten} e.g. a slow general decay of sensitisation.
\end{enumerate}
\end{block}  	
\end{frame}
%%--------------------------------------------------
\begin{frame}[t]{AFM as a Bipartite Network}
  \begin{columns}
\column{0.5\textwidth}
\begin{figure}
\centering
\includegraphics[height=0.65\textwidth, angle=0]
{/media/Preload/Pub2010/ThoughtsLinedUp/images/AFM-Diag2.eps}
%figure caption is below the figur
\caption{\scriptsize The attractive filed model (AFM)}
\label{fig:afm} % Give a unique label
\end{figure}
\column{0.55\textwidth}
\begin{scriptsize}
%\scriptsize \alert{AFM as a bipartite nework}
      \begin{tabular}{p{0.85in}|p{1.2in}}
      \hline
      Source nodes (o) & \scriptsize tasks to be allocated to agents\\
      \hline
      Agent nodes (x) & \scriptsize E.g., ants, humans, or robots\\
     \hline
     Black solid edges & \scriptsize attractive fields that correspond to an agent's perceived stimuli from each task\\
	\hline
	Green edges &  attractive fields of no-task option shown as a particular task (w)\\
	\hline
	Red lines & not edges, but represent how each agent is allocated to a single task.\\% at any point in time\\
	\hline
      \end{tabular}
\end {scriptsize}
\end{columns}
\end{frame}
%--------------------------------------------------------
\begin{frame}[t]{Properties of Agents under AFM}	
\alert{The probability of an agent choosing to perform a task:}
\begin{equation}
P_{j}^{i} = \frac{S_{j}^{i}}{\sum_{j=0}^{J} S_{j}^{i}} \hspace*{0.25cm}where,\hspace*{0.25cm}S^{i}_{0} = S^{i}_{RW}   
\label{eqn:afm3}
\end{equation}
\alert{The strength of an attractive field} varies according to the \alert{sensitivity} of the agent is to that task, $k_{j}^{i}$, the \alert{distance} between the task and the agent, $d_{ij}$, and the \alert{\em urgency}, $\phi _{j}$ of the task.
\begin{equation}
S_{j}^{i} = tanh\{\frac{k_{j}^{i}}{d_{ij}+\delta } \phi _{j}\}
\label{eqn:afm1}
\end{equation} 
\small {\em Delta distance} $\delta$, is a small constant, to avoid division by zero, in the case when a robot has reached to a task.
\end{frame}

%%--------------------------------------------------
\begin{frame}[t]{AFM and Self-organization}
  \begin{itemize}
    \item \alert{Positive feedback} through learning\\ Example: Increasing task-sensitization of agents
\begin{equation}
 If\hspace*{0.15cm}task\hspace*{0.15cm}is\hspace*{0.15cm}done:\hspace*{0.15cm}  k^i_j \rightarrow   k^i_j \hspace*{0.15cm} + \hspace*{0.15cm} k_{INC}
\label{eqn:k-inc}
\end{equation} 
    \item \alert{Negative feedback} through forgetting\\ Exampel: Decreasing task-sensitization of agents
\begin{equation}
 If\hspace*{0.15cm}task\hspace*{0.15cm}is\hspace*{0.15cm}not\hspace*{0.15cm}done:\hspace*{0.15cm}  k^i_j \rightarrow   k^i_j \hspace*{0.15cm} - \hspace*{0.15cm} k_{DEC}
\label{eqn:k-dec}
\end{equation}   	
	\item \alert{Multiple interactions} through continuous flow of information.
	
	\item \alert{Randomness} through stochastic task-selection.
    \end{itemize}
\end{frame}
%%--------------------------------------------------
\begin{frame}[t]{Related issues for using AFM in real-world application}
\begin{columns}
\column{0.6\textwidth}
\begin{figure}
\centering
\includegraphics[height=0.6\textwidth, angle=0]
{/media/Preload/Pub2010/RAS-Draft/images/RILCamcorderSnapshot1.eps}
%figure caption is below the figur
\caption{\scriptsize Modelling real-world application to a laboratory scenario}
\label{fig:self-org-agent} % Give a unique label
\end{figure}
\column{0.53\textwidth}
\begin{block}{\small Map tasks \& robot capabilities}
\begin{enumerate}
\item \scriptsize workload $\Leftrightarrow$ task-urgency
\item \scriptsize work done $ \Leftrightarrow$ task-urgency decrease \item \scriptsize work pending $ \Leftrightarrow$ task-urgency increase
\end{enumerate}
\end{block}
\begin{block}{\small Enable continuous flow of info}
\begin{enumerate}
\item \scriptsize Centralized communication 
\item \scriptsize Local communication
\item \scriptsize Stigmergic communication
\end{enumerate}
\end{block}
\end{columns}  
\end{frame}	
%-----------------------------------------------------------
\begin{frame}[t]{A Manufacturing Shop-Floor Interpretation of AFM}
\begin{columns}
\column{0.6\textwidth}
\begin{figure}
\centering
\includegraphics[height=0.6\textwidth, angle=0]
{/media/Preload/Pub2010/RAS-Draft/images/VSP.eps}
%figure caption is below the figur
\caption{\scriptsize Production and maintenance cycle of a manufacturing shop-floor}
\label{fig:vsp} % Give a unique label
\end{figure}
\column{0.53\textwidth}
\begin{scriptsize}
      \begin{tabular}{p{0.7in}|p{1.2in}}
      \hline
      \alert{Initial task \protect\newline urgency} & \scriptsize workload x $ \delta \phi_{INC}$\\
      \hline
      \alert{If task \protect\newline unattended} & \scriptsize work-load increases by $\delta \phi_{INC}$\\
     \hline
     \alert{If task served} & \scriptsize work-load decreases by $\delta \phi_{DEC}$\\
	\hline
	\alert{Average \protect\newline Production Completion Delay (APCD)} &  (\textit{Ideal production time} - \textit{Actual production time})/  \textit{Ideal production time} \\
	\hline
	\alert{Average \protect\newline Pending Maintenance Work (APMW)} & (\textit{Total pending maintenance work in all machines})/\textit{Total no. of machines.}\\
	\hline
      \end{tabular}
\end {scriptsize}
\end{columns}  
\end{frame}	
%%%===========================================================
\section{Communication Models}
\begin{frame}[t]{Centralized and Local Communication Models}
\begin{columns}
\column{0.6\textwidth}
\begin{figure}
\centering
\includegraphics[height=0.6\textwidth, angle=0]
{/media/Preload/Pub2010/RAS-Draft/images/CentralizedComm.eps}
%figure caption is below the figur
\caption{\scriptsize A centralized communication scheme}
\label{fig:vsp} % Give a unique label
\end{figure}
\column{0.6\textwidth}
\begin{scriptsize}
      \begin{tabular}{p{0.9in}|p{1in}}
      \hline
      \alert{Centralized Model} & \alert{Local Model}\\
      \hline
      Global \alert{broadcast} messaging & Local \alert{peer-to-peer} messaging\\
      \hline
      Communicate synchronously & Communicate when peer(s) come in close contact (inside range $r_{comm}$)\\
      \hline
      Modelled after \alert{\textit{Polistes} wasps}:
      global sensing no peer-to-peer communication & Modelled after \protect\newline \alert{\textit{Polybia} wasps}: \protect\newline
      local sensing local communication\\	  	
	  \hline
      \end{tabular}
\end {scriptsize}
\end{columns}  
\end{frame}	
%%%===========================================================
\section{Implementation}
\begin{frame}[t]{Hybrid-event Driven Architecture on D-Bus}
\begin{figure}
\centering
\includegraphics[width=0.9\textwidth, angle=0]
{/media/Preload/Pub2010/RAS-Draft/images/RIL-Expt-Setup1.eps}
%figure caption is below the figure
\caption{\scriptsize Hardware and software setup for centralized communication experiments}
\label{fig:RIL-Expt-Setup1} % Give a unique label
\end{figure}
\end{frame}
%---------------------------------------------------
\begin{frame}[t]{Tracking e-puck robots}
\begin{columns}
\column{0.8\linewidth}
\begin{figure}
\centering
\subfigure[SwisTrack multi-robot tracker]
{
\includegraphics[width=0.85\linewidth]
{/media/Preload/Pub2010/RAS-Draft/images/SwisTrackScreenshot.eps}
}
\end{figure}
\column{0.25\linewidth}
\begin{figure}
\subfigure[E-puck robot]{
\includegraphics[width=2.8cm, height=2.5cm]{/media/Preload/Pub2010/RAS-Draft/images/epuck-happy.eps}
}
\label{fig:e-puck}
\end{figure}
%%
\begin{figure}
\subfigure[E-puck marker]{
\includegraphics[width=2.5cm, height=1.8cm]{/media/Preload/Pub2010/RAS-Draft/images/20-31412.eps}
}
\label{fig:e-puck}
\end{figure}
\end{columns}
\end{frame}
%%%===========================================================
\section{Results}
%---------------------------------------------------------------
\begin{frame}[t]{Results: Shop-floor Work-load and Active Workers }
\begin{columns}
\column{0.5\linewidth}
\begin{figure}
\centering
%\subfigure[\small Changes in task-urgency (CCM)]{
\includegraphics[width=0.85\linewidth]
{/media/Preload/Pub2010/RAS-Draft/images/SB-PlotUrgencyLog-2010May10-115549.eps}
%}
\caption{\scriptsize Changes in task-urgency}
\end{figure}
%%
\vspace*{-0.5cm}
\begin{scriptsize}
\alert{Shop-floor work-load:}\\
Sum of changes in task-urgencies of\\ all $M$ tasks at $(q+1)^{th}$ step:
\begin{equation} 
\Delta \Phi_{j, q+1} = \sum_{j=1}^{M} (\phi_{j, q+1} - \phi_{j, q})
\label{eqn:Delta-Phi}
\end{equation}
\end{scriptsize}
%%
\column{0.5\linewidth}
\vspace*{-1.2cm}
\begin{figure}
%\subfigure[\small Shop-floor work-load]{
\includegraphics[width=0.85\linewidth]{/media/Preload/Pub2010/RAS-Draft/images/SB-TaskUrgencyStat.eps}
\caption{\scriptsize Shop-floor work-load}
\label{fig:r2}
\end{figure}
\begin{scriptsize}
\vspace*{-0.45cm}
\alert{Active worker ratio:}\\
\begin{equation} 
\frac{\textit{Active workers in all tasks}}{\textit{Total available workers}}
\label{eqn:Delta-Phi}
\end{equation}
\end{scriptsize}
%%
\end{columns}
\end{frame}
%--------------------------------------------------------------
\begin{frame}[t]{Results: Shop-floor Work-load and Active Workers }
\begin{columns}
\column{0.5\linewidth}
\vspace*{-0.8cm}
\begin{figure}
\centering
%\subfigure[\small Changes in task-urgency (CCM)]{
\includegraphics[width=0.7\linewidth]
{/media/Preload/Pub2010/RAS-Draft/images/SB-TaskUrgencyStat.eps}
%}
\caption{\scriptsize Shop-floor workload under CCM}
\end{figure}
%%
\vspace*{-1cm}
\begin{figure}
\centering
%\subfigure[\small Changes in task-urgency (CCM)]{
\includegraphics[width=0.7\linewidth]
{/media/Preload/Pub2010/RAS-Draft/images/SB-WorkerRatio.eps}
%}
\caption{\scriptsize Active worker ratio under CCM}
\end{figure}
%%
\column{0.5\linewidth}
\vspace*{-0.8cm}
\begin{figure}
\includegraphics[width=0.7\linewidth]{/media/Preload/Pub2010/RAS-Draft/images/SD-TaskUrgencyStat.eps}
\caption{\scriptsize Shop-floor work-load under LCM}
\end{figure}
\vspace*{-1cm}
\begin{figure}
\includegraphics[width=0.7\linewidth]{/media/Preload/Pub2010/RAS-Draft/images/SD-Local1m-Plasticity.eps}
\caption{\scriptsize Active worker ratio under LCM}
\end{figure}
\end{columns}
\end{frame}
%---------------------------------------------------------------
\begin{frame}[t]{Results: Task-Performance}
\begin{table}
\begin{small}
\begin{center}
\caption{Shop-floor production and maintenance task performance}
\begin{tabular}{|p{2in}|c|c|}
\hline \textbf{Series} & \textbf{APCD} & \textbf{APMW} (time-step) \\ 
\hline \alert{A.} Centralized communication, \protect\newline 8 robots, 2 tasks, 2 $m^2$ area & 1.22 & 1\\ 
\hline \alert{B.} Centralized communication, \protect\newline 16 robots, 4 tasks, 4 $m^2$ area & 2.3 & 3\\
\hline \alert{C.} Local communication, \protect\newline 16 robots, 4 tasks, 4 $m^2$ area, $r_{comm}$=0.5m & 1.42 & 5\\
\hline \alert{D.} Local communication, \protect\newline 16 robots, 4 tasks, 4 $m^2$ area, $r_{comm}$=1m  & 1.46 & 2\\
\hline
\end{tabular}
\label{table:motion-cmp} 
\end{center}
\end{small}
\end{table}
\end{frame}
%---------------------------------------------------------------
\begin{frame}[t]{Results: Task-specialization}
\begin{small}
\alert{Overall group task-specialization} in terms of peak values of sensitization of all robots:
\vspace*{-0.25cm}
\begin{equation}
K^G_{avg} = \frac{1}{N}\sum_{i=1}^{N} \max_{j=1}^M\left ( k^i_{j, q} \right ) 
\label{eqn:K-G}
\end{equation}
Time-step values ($q$) taken to reach those peak values for all robots:
%%
\vspace*{-0.25cm}
\begin{equation}
Q^G_{avg}= \frac{1}{N}\sum_{i=1}^{N} q^i_{k=k_{max}}
\label{eqn:Q-G}
\end{equation}
\end{small}
%%
\vspace*{-0.75cm}
\begin{table}
\begin{small}
\begin{center}
\caption{Task-specialization values of the robots}
\begin{tabular}{|c|c|c|}
\hline Series & \textbf{$ K^G_{avg}$} (SD) & \textbf{$ Q^G_{avg}$} (SD) \\ 
\hline \alert{A} & \alert{0.40} (0.08)& \alert{38} (13)\\ 
\hline B & 0.30 (0.03) & 18 (5)\\
\hline \alert{C}  & \alert{0.39} (0.17) & \alert{13} (7)\\
\hline D  & 0.27 (0.1) & 11 (5)\\
\hline
\end{tabular}
\label{table:motion-cmp} 
\end{center}
\end{small}
\end{table}
\end{frame}
%---------------------------------------------------------------
\begin{frame}[t]{Results: Energy-usage}
\begin{table}
\begin{small}
\begin{center}
\caption{Sum of translations of robots in our experiments.}
\begin{tabular}{|p{2in}|c|c|}
\hline \textbf{Series} & \textbf{Average translation (m)} & \textbf{SD} \\ 
\hline \alert{A.} Centralized communication, \protect\newline 8 robots, 2 tasks, 2 $m^2$ area & 2.631 & 0.804\\ 
\hline \alert{B.} Centralized communication, \protect\newline 16 robots, 4 tasks, 4 $m^2$ area & \alert{13.882} & 3.099\\
\hline \alert{C.} Local communication, \protect\newline 16 robots, 4 tasks, 4 $m^2$ area, $r_{comm}$=0.5m & \alert{4.907} & 1.678\\
\hline \alert{D.} Local communication, \protect\newline 16 robots, 4 tasks, 4 $m^2$ area, $r_{comm}$=1m  & 4.854 & 1.592\\
\hline
\end{tabular}
\label{table:motion-cmp} 
\end{center}
\end{small}
\end{table}
\end{frame}
%---------------------------------------------------------------
\begin{frame}[t]{Results: Communication Loads}
\begin{columns}
\column{0.5\linewidth}
\vspace*{-0.8cm}
\begin{figure}
\centering
%\subfigure[\small Changes in task-urgency (CCM)]{
\includegraphics[width=0.7\linewidth]
{/media/Preload/Pub2010/RAS-Draft/images/SA-8Robot-SignalingFreqStat.eps}
%}
\caption{\scriptsize Frequency of \texttt{TaskInfo} signalling under Series A}
\end{figure}
%%
\vspace*{-1cm}
\begin{figure}
\centering
%\subfigure[\small Changes in task-urgency (CCM)]{
\includegraphics[width=0.7\linewidth]
{/media/Preload/Pub2010/RAS-Draft/images/SB-SignalingFreqStat.eps}
%}
\caption{\scriptsize Frequency of \texttt{TaskInfo} signalling under Series B}
\end{figure}
%%
\column{0.5\linewidth}
\vspace*{-0.8cm}
\begin{figure}
\includegraphics[width=0.7\linewidth]{/media/Preload/Pub2010/RAS-Draft/images/SC-Local-500cm-SignalingFreqStat.eps}
\caption{\scriptsize Frequency of \texttt{TaskInfo} signalling under Series C}
\end{figure}
\vspace*{-1cm}
\begin{figure}
\includegraphics[width=0.7\linewidth]{/media/Preload/Pub2010/RAS-Draft/images/SD-Local-1m-SignalingFreqStat.eps}
\caption{\scriptsize Frequency of \texttt{TaskInfo} signalling under Series D}
\end{figure}
\end{columns}
\end{frame}

%%%===========================================================
\section{Conclusions}
%%-----------------------------------
\begin{frame}[t]{Conclusions}
\begin{itemize}
    \item \normalsize \alert{AFM solves the MRTA issue for a relatively large group}\\
    $\rightarrow$ \small under both centralized and local communication strategies.
    \item \normalsize \alert{Task-performance varies under different strategies}\\
    $\rightarrow$ \small for small group, task-performance degrades in centralized communication\\
    $\rightarrow$ \small for large group, local  communication increases task-specialization and significantly reduces motions.    
    \item \normalsize \alert{AFM can model complex multi-tasking environment}\\
    $\rightarrow$ \small such as a dynamic manufacturing shop-floor.
    \item \normalsize \alert{Maximizing information flow is not useful}\\
    $\rightarrow$ \small  under a stochastic task-allocation process, more information tends to cause more task-switching behaviours. % that lowers the level of task-specialization.     
 \end{itemize}
\end{frame}
%%%===========================================================
\section{Contributions}
\begin{frame}[t]{General Contributions}
%
\begin{itemize}
    \item \normalsize \alert{Self-organization in artificial systems}\\ 
$\rightarrow$ \small Self-organized allocation produces specialized workers even when the group size is \textit{small} ($<$ 10).
    \item \normalsize \alert{Role of communication in self-organization}\\
$\rightarrow$ \small  Local communication in task-allocation outperforms centralized one in terms of group level task-specialization and energy usage.
\item \normalsize \alert{Large-scale system development}\\
$\rightarrow$ \small Bottom-up de-coupled construction of \textit{large} artificial system yields higher advantages particularly, flexibility and integration with inter-operable elements.%in achieving self-regulated MRTA.
\end{itemize}
%
\end{frame}
%
\begin{frame}[t]{Specific Contributions}
\begin{itemize}
    \item \alert{Interpreted AFM}\\ 
$\rightarrow$ \small as a basic mechanism for multi-robot task-allocation%, specifically for abstract task-allocation in mult-tasking environment.
    \item \normalsize \alert{Validated the effectiveness of AFM}\\
$\rightarrow$ \small with reasonably \textit{large} number of real robots % i.e., 16 e-puck robots.
\item \normalsize \alert{Compared the performances of two communication and sensing strategies:}%in achieving self-regulated MRTA.
\begin{enumerate}
\begin{scriptsize}
\item Centralized communication like \textbf{Polistes} wasps
\item Local communication like \textbf{Polybia} wasps
\end{scriptsize}
\end{enumerate}
\item \normalsize \alert{Developed a \textit{flexible} multi-robot control architecture}\\ 
$\rightarrow$ \small using \textbf{D-Bus} inter-process communication % technology.
\item \normalsize \alert{Classified MRTA solutions focusing three major issues:}\\ 
\begin{small}
%\small Focusing 3 major issues:\\
%\hspace*{2mm} 
\begin{enumerate}
\begin{scriptsize}
\item Organization of task-allocation
\item Communication and
\item Interaction
\end{scriptsize}
\end{enumerate}
\end{small}
\end{itemize}
\end{frame}
%%%===========================================================
\section{Future works}
\begin{frame}[t]{Future works}
  \begin{itemize}
    \item \normalsize \alert{Deploying our task-allocation model in various task settings}\\
    $\rightarrow$ \small e.g. dynamic tasks, co-operative tasks, heterogeneous tasks.
    \item \normalsize \alert{Find optimum communication range}\\
    $\rightarrow$ \small as a property of self-regulation of individuals.
    \item \normalsize \alert{Real-world implementation}\\
    $\rightarrow$ \small e.g. warehouse automation, manufacturing shop-floor or any other multi-tasking environment.
    \end{itemize}
\end{frame}
%%%%%%%%%%%%%%
\end{document}


