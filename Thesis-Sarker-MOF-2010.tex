% Setup appearance:
%\documentclass[handout,draft]{beamer}
\documentclass{beamer}
\usetheme{Frankfurt} % right index Marburg
\usefonttheme[onlylarge]{structurebold}
\setbeamerfont*{frametitle}{size=\normalsize,series=\bfseries}
\setbeamertemplate{navigation symbols}{}
% Standard packages
\usepackage[english]{babel}
\usepackage[latin1]{inputenc}
\usepackage{times}
\usepackage[T1]{fontenc}
\usepackage{makeidx} % allows for indexgeneration
\usepackage{graphicx}
\usepackage{multicol}
\usepackage{subfigure}
\usepackage{mathptmx} % use Times fonts if available on your TeX system
\usepackage{setspace}
\usepackage{epsfig}
\usepackage{array}
% Setup TikZ
\usepackage{tikz}
\usetikzlibrary{arrows}
\tikzstyle{block}=[draw opacity=0.7,line width=1.4cm]

% Author, Title, etc.
\title[Self-regulated Multi-robot Task~Allocation] 
{%
  Self-regulated Multi-robot Task~Allocation%
}

\author[MOFSarker]
{
  Md Omar Faruque Sarker
}

\institute[UWN]
{
 PhD Student\\
 Cognitive Robotics Research Centre\\
 Newport Business School\\
 University of Wales, Newport
}

\date{14 October 2010}



% The main document

\begin{document}

\begin{frame}
  \titlepage
\end{frame}

\begin{frame}{Outline}
  \tableofcontents
\end{frame}
%%%===========================================================
\section{Introduction}
\begin{frame}[t]{Background: The EPSRC Project: ``Defying the rules - How Self Regulatory Social Systems Work''}
\begin{block}{Objectives}
\begin{itemize}
\item \alert{Identify generic rules} that allow social systems to develop sustainability through \alert{self-regulation}.
\item Improve the \alert{performance} and \alert{robustness} in the organization of social systems. 
\end{itemize}
\end{block}
\begin{block}{Our collaborators}
\begin{itemize}
\item The Applied Mathematics Research Group,\\ \alert{University of West of England}
\item The Centre for Systems Studies,\\ \alert{University of Hull}
\item The Condensed Matter Theory Group,\\ \alert{ Imperial College, London}
\end{itemize}
\end{block}
\end{frame}
%%--------------------------------------------------
\begin{frame}[t]{Multi-robot Task Allocation (MRTA)}
	
\begin{block}{What is MRTA?}
In a multi-tasking environment \alert{dynamically allocate appropriate tasks to appropriate robots} considering the  changes in task-requirements, team-performance and environment.
\end{block}
  	
\begin{block}{Why MRTA is difficult?}
In typical large distributed multi-robot teams:
\begin{itemize}
\item No centralized planner or coordinator
\item \alert{Robots have limited ability}\\
$\rightarrow$ \small to sense, communicate and interact locally
\item \normalsize \alert{Robots have limited world-views}\\ 
$\rightarrow$ \small knowledge of past, present and future actions of others
\end{itemize}
\end{block}
  	
\end{frame}
%%--------------------------------------------------
\begin{frame}[t]{Major Approaches for MRTA}
\vspace*{-0.4cm}
 \begin{columns}
      \column{.5\textwidth}	
\begin{block}{Explicit allocation}
Through \alert{explicit modelling} of environment, tasks, robot capabilities. Some forms are: \alert{knowledge based, market based, role based, control theoretic}. 
\begin{itemize}
\item \small Pros: Straight-forward to design, implement and analyse formally\footnote{\scriptsize Parker, \textit{J. of Phy. Agents}, 2:2, 2008}. % predictable
\item \small Cons: \alert{Not suitable for large teams ($>$ 10)}, heavy dependency on explicit global broadcast communication \footnote{\scriptsize Lerman \textit{et. al}, \textit{IJRR} 25, 2006}.
\end{itemize}
\end{block}
 \column{.5\textwidth}
\begin{block}{Self-organized allocation}
Through \alert{emergent group behaviour} produced by the local interaction and implicit or local communication. Most common form is: \alert{response threshold} based approach.
\begin{itemize}
\item \small Pros: Suitable for large teams, no explicit model, implicit/local communication\footnote{\scriptsize Pugh \& Martinoli, \textit{Swarm Intell:}3, 2009}.
\item \small Cons: \alert{Difficult to design, implement, analyse and limited to one specific global task.\footnote{\scriptsize Gerkey \& Mataric, \textit{IJRR}, 23, 2004}.} % unpredictable
\end{itemize}
\end{block}
\end{columns}
\end{frame}
%%--------------------------------------------------
\begin{frame}[t]{Self-organization}
\begin{block}{What is Self-organization?}
\small \alert{Pattern formation} in both biological and physical systems through the local interactions internal to the system \footnote{\scriptsize Camazine \textit{et al.}, Self-organization in Biological Systems, 2001.}.%\\
%$\rightarrow$ \small global patterns from local interactions.
\end{block}
\begin{columns}
\column{0.5\textwidth}
%
\begin{block}{Ingredients}
\begin{itemize}
\item \small \alert{Positive feedback}\\
$\rightarrow$ \scriptsize ants' recruitments to food source.
\item \small \alert{Negative feedback}\\  
$\rightarrow$ \scriptsize overcrowding at food sources.
\item  \small \alert{Multiple interactions}\\ 
$\rightarrow$ \scriptsize peer-to-peer, broadcast communication
\item \small \alert{Randomness}\\
$\rightarrow$ \scriptsize error in trail-following
\end{itemize}
\end{block}
%%
\column{0.45\textwidth}
\begin{block}{Why Self-organized approach?}
\begin{itemize}
\item  \small Implementing simple agent behaviours is \alert{economical}%\\
%$\rightarrow$ \scriptsize no sophisticated cognitive agents.
\item  \small \alert{Easily scalable} for large robot-teams and tasks%\\  
%$\rightarrow$ \scriptsize no explicit modelling of environment.
\item  \small \alert{Fault-tolerant}%\\ 
%$\rightarrow$ \scriptsize no leaders, templates or blue-prints.
\item \small \alert{Energy-efficient}%\\
%$\rightarrow$ \scriptsize no costly communication or computation overhead.
\end{itemize}
\end{block}
\end{columns}
\end{frame}
%%%--------------------------------------------------
%\begin{frame}[t]{Ingredients of Self-organization}
%\begin{columns}
%\column{0.7\textwidth}
%\begin{figure}
%\centering
%\includegraphics[height=0.7\textwidth, angle=0]
%{/media/Preload/Pub2010/ThoughtsLinedUp/images/dia-files/self-org-1}
%%figure caption is below the figur
%\caption{\scriptsize The four ingredients of self-organization}
%\label{fig:self-org} % Give a unique label
%\end{figure}
%\column{0.3\textwidth}
%\scriptsize Examples:
%      \begin{tabular}{l|p{1in}}
%      \hline
%      A & \scriptsize \alert{Ants' recruitments to food source} through trail laying/following\\
%      \hline
%     B & \scriptsize \alert{Overcrowding} at food sources\\
%     \hline
%     C & \scriptsize \alert{Various types of communications} through peer-to-peer, broadcast or stigmergic\\
%     \hline
%     D & \scriptsize \alert{Randomness} and/or error in trail-following that leads to discover new food sources\\
%     \hline       
%      \end{tabular}
%
%\end{columns}
%\end{frame}
%--------------------------------------------
%\begin{frame}[t]{Self-regulation of an Agent}
%
%\begin{columns}
%\column{0.5\textwidth}
%\begin{figure}
%\centering
%\includegraphics[height=0.5\textwidth, angle=0]
%{/media/Preload/Pub2010/ThoughtsLinedUp/images/dia-files/self-org-agent}
%%figure caption is below the figur
%\caption{\scriptsize Three major interfaces of a self-organized agent}
%\label{fig:self-org-agent} % Give a unique label
%\end{figure}
%\column{0.5\textwidth}
%\scriptsize \alert{Self-organization in birds nesting}
%      \begin{tabular}{p{0.7in}|p{1.2in}}
%      \hline
%      Simple \protect\newline behavioural rules & \scriptsize \alert{Follow: \textit{``I nest close where you nest \protect\newline ... \protect\newline unless overcrowded''}} \\
%      \hline
%     Local \protect\newline communication & \scriptsize \alert{Communications through local broadcast signals}\\
%     \hline
%     Local \protect\newline interactions & \scriptsize \alert{Courtship display with neighbours}\\
%	\hline
%      \end{tabular}
%
%\end{columns}
%\end{frame}
%%--------------------------------------------------
%%%===========================================================
\section{Task Allocation by Attractive Field Model (AFM)}
%%--------------------------------------------------
\begin{frame}[t]{Attractive Field Model (AFM)}
\vspace{-0.5cm}	
\begin{block}{Features of AFM}
\begin{itemize}
\item \alert{Interdisciplinary:} \small Developed from the study of ant, human and robotic social systems\footnote{\scriptsize Arcaute \textit{et al.} Ecol. Complexity, 6:4 2008.}. 
%Division of labour in ant colonies in terms of attractive fields
\item \normalsize \alert{Abstract:} \small Sufficiently abstract to accommodate different sensing and communication models.
\end{itemize}
\end{block}
  	
\begin{block}{Requirements of Self-regulation}
\begin{enumerate}
\item \small  \alert{Concurrence:} ``The simultaneous presence of several tasks''\\ 
$\rightarrow$ \scriptsize at least a single task and the option of not doing any task. 
\item \small  \alert{Continuous flow of information:}\\ 
$\rightarrow$ \scriptsize to perceive tasks and receive feedback on system performance.
\item \small \alert{Sensitization:} ``Individuals having different levels of preference''\\ 
$\rightarrow$ \scriptsize to all available tasks.
\item \small  \alert{Forgetting:} ``A mechanism to reduce sensitisation levels''\\ 
$\rightarrow$ \scriptsize e.g. a slow general decay of sensitisation.
\end{enumerate}
\end{block}  	
\end{frame}
%%--------------------------------------------------
\begin{frame}[t]{AFM as a Bipartite Network}
  \begin{columns}
\column{0.5\textwidth}
\vspace*{-0.25cm}
\begin{figure}
\centering
\includegraphics[height=0.65\textwidth, angle=0]
{/media/Preload/Pub2010/ThoughtsLinedUp/dia-files/AFM-Diag3.eps}
%figure caption is below the figur
\caption{\scriptsize The attractive filed model (AFM)}
\label{fig:afm} % Give a unique label
\end{figure}
\column{0.55\textwidth}
\vspace*{-0.25cm}
\begin{scriptsize}
%\scriptsize \alert{AFM as a bipartite nework}
      \begin{tabular}{p{0.85in}|p{1.3in}}
      \hline
      Source nodes (o) & \scriptsize \alert{tasks} to be allocated\\      \hline
      Agent nodes (x) & \scriptsize \alert{agents} e.g., ants, humans, or robots\\
     \hline
     Black solid edges & \scriptsize \alert{attractive fields} that correspond to an agent's perceived stimuli from each \alert{task}\\
	\hline
	Green edges &  \alert{attractive fields of no-task} option shown as  task (w)\\
	\hline
	Black dashed edges & not edges, but shows an agent allocated to a task.\\% at any point in time\\
	\hline
      \end{tabular}
\end {scriptsize}
\end{columns}
%%
\begin{columns}
\column{0.55\textwidth}
%\vspace*{-0.25cm}
\small \alert{Agent's probability to choose a task:}
%\vspace*{-0.5cm}
\begin{equation}
\scriptsize
P_{j}^{i} = \frac{S_{j}^{i}}{\sum_{j=0}^{J} S_{j}^{i}} \hspace*{0.25cm}where,\hspace*{0.25cm}S^{i}_{0} = S^{i}_{RW}   
\label{eqn:afm3}
\vspace*{-0.15cm}
\end{equation}
\scriptsize \alert{$S_{j}^{i}$ and $S^{i}_{RW}$:}  $i$ agent's \alert{stimuli to $j$ task} and \alert{random-walk.}
\vspace*{0.15cm}
\column{0.5\textwidth}
\small \alert{Strength of an attractive field:} 
\begin{equation}
\scriptsize
S_{j}^{i} = tanh\{\frac{k_{j}^{i}}{d_{ij}+\delta } \phi _{j}\}
\label{eqn:afm1}
\end{equation}
%\vspace*{-0.25cm}
\scriptsize
\alert{$k_{j}^{i}$, $d_{ij}$:}   $i$ agent's \alert{sensitization} and \alert{distance} to task $j$. \alert{$\phi _{j}$:} \alert{urgency} of task $j$. 
\end{columns}
\end{frame}
%--------------------------------------------------------
%\begin{frame}[t]{Properties of Agents under AFM}	
%\alert{The probability of an agent choosing to perform a task:}
%\begin{equation}
%P_{j}^{i} = \frac{S_{j}^{i}}{\sum_{j=0}^{J} S_{j}^{i}} \hspace*{0.25cm}where,\hspace*{0.25cm}S^{i}_{0} = S^{i}_{RW}   
%\label{eqn:afm3}
%\end{equation}
%\alert{The strength of an attractive field} varies according to the \alert{sensitivity} of the agent is to that task, $k_{j}^{i}$, the \alert{distance} between the task and the agent, $d_{ij}$, and the \alert{\em urgency}, $\phi _{j}$ of the task.
%\begin{equation}
%S_{j}^{i} = tanh\{\frac{k_{j}^{i}}{d_{ij}+\delta } \phi _{j}\}
%\label{eqn:afm1}
%\end{equation} 
%\small {\em Delta distance} $\delta$, is a small constant, to avoid division by zero, in the case when a robot has reached to a task.
%\end{frame}

%%--------------------------------------------------
\begin{frame}[t]{AFM and Self-organization}
  \begin{itemize}
    \item \alert{Positive feedback} through learning\\ Example: Increasing task-sensitization of agents\\
With  an agent's \textit{rate of learning} tasks, $k_{INC}$:
\begin{equation}
\alert{ 
 If\hspace*{0.15cm}task\hspace*{0.15cm}is\hspace*{0.15cm}done:\hspace*{0.15cm}  k^i_j \rightarrow   k^i_j \hspace*{0.15cm} + \hspace*{0.15cm} k_{INC}
 }
\label{eqn:k-inc}
\end{equation} 
    \item \alert{Negative feedback} through forgetting\\ Example: Decreasing task-sensitization of agents\\
    With an agent's \textit{rate of forgetting} tasks, $k_{DEC}$:
\begin{equation}
\alert{
 If\hspace*{0.15cm}task\hspace*{0.15cm}is\hspace*{0.15cm}not\hspace*{0.15cm}done:\hspace*{0.15cm}  k^i_j \rightarrow   k^i_j \hspace*{0.15cm} - \hspace*{0.15cm} k_{DEC}
 }
\label{eqn:k-dec}
\end{equation}   	
	\item \alert{Multiple interactions} through continuous flow of information.
	
	\item \alert{Randomness} through stochastic task-selection.
    \end{itemize}
\end{frame}
%%--------------------------------------------------
\begin{frame}[t]{Related issues for using AFM in real-world application}
\begin{columns}
\column{0.6\textwidth}
\begin{figure}
\centering
\includegraphics[height=0.6\textwidth, angle=0]
{/media/Preload/Pub2010/RAS-Draft/images/RILCamcorderSnapshot1.eps}
%figure caption is below the figur
\caption{\scriptsize Modelling real-world application to a laboratory scenario}
\label{fig:self-org-agent} % Give a unique label
\end{figure}
\column{0.53\textwidth}
\begin{block}{\small Map tasks \& robot capabilities}
\begin{enumerate}
\item \scriptsize workload $\Leftrightarrow$ task-urgency
\item \scriptsize work done $ \Leftrightarrow$ task-urgency decrease \item \scriptsize work pending $ \Leftrightarrow$ task-urgency increase
\end{enumerate}
\end{block}
\begin{block}{\small Enable continuous flow of info}
\begin{enumerate}
\item \scriptsize Centralized communication 
\item \scriptsize Local communication
\item \scriptsize Stigmergic communication
\end{enumerate}
\end{block}
\begin{block}{\small Other issues}
\begin{enumerate}
\item \scriptsize Enable learning/forgetting in controller
\item \scriptsize Perception of distance $\Leftrightarrow$ localization
%\item \scriptsize Provide multiple tasks (include random-walk)
\end{enumerate}
\end{block}
\end{columns}  
\end{frame}	
%-----------------------------------------------------------
\begin{frame}[t]{A Manufacturing Shop-Floor Interpretation of AFM}
\begin{columns}
\column{0.6\textwidth}
\begin{figure}
\centering
\includegraphics[height=0.6\textwidth, angle=0]
{/media/Preload/Pub2010/RAS-Draft/images/VSP.eps}
%figure caption is below the figur
\caption{\scriptsize Production and maintenance cycles of a manufacturing shop-floor}
\label{fig:vsp} % Give a unique label
\end{figure}
\column{0.53\textwidth}
\scriptsize AFM validation under a shop-floor scenario\footnote{\scriptsize Sarker \& Dahl. \textit{LNCS} 6234, 2010.}.
\begin{scriptsize}
      \begin{tabular}{m{0.7in}|m{1.2in}}
      \hline
      \alert{Initial task \protect\newline urgency} & \scriptsize workload x $ \delta \phi_{INC}$\\
      \hline
      \alert{If task \protect\newline unattended} & \scriptsize work-load increases by $\delta \phi_{INC}$\\
     \hline
     \alert{If task served} & \scriptsize work-load decreases by $\delta \phi_{DEC}$\\
	\hline
	\alert{Average \protect\newline Production Completion Delay (APCD)} &  (\textit{Ideal production time} - \textit{Actual production time})/  \textit{Ideal production time} \\
	\hline
	\alert{Average \protect\newline Pending Maintenance Work (APMW)} & (\textit{Total pending maintenance work in all machines})/\textit{Total no. of machines.}\\
	\hline
      \end{tabular}
\end {scriptsize}
\end{columns}  
\end{frame}	
%%%===========================================================
\section{Communication Models}
\begin{frame}[t]{Centralized and Local Communication Models}
\begin{columns}
\column{0.6\textwidth}
\begin{figure}
\centering
\includegraphics[height=0.6\textwidth, angle=0]
{/media/Preload/Pub2010/RAS-Draft/images/CentralizedComm.eps}
%figure caption is below the figur
\caption{\scriptsize A centralized communication scheme}
\label{fig:vsp} % Give a unique label
\end{figure}
\column{0.6\textwidth} 
\scriptsize \alert{Communication models inspired by wasps}\footnote{\scriptsize Jeanne. \textit{Info. process. in social insects}, 1999}
\vspace*{0.1cm}
\begin{scriptsize}
      \begin{tabular}{m{0.95in}|m{1in}}
      \hline
      \textbf{Centralized Model} & \textbf{Local Model}\\
      \hline
      \includegraphics[width=2.5cm, height=1.5cm]{/media/Preload/Pub2010/ThoughtsLinedUp/photos/Wasps_wikimedia.org_Polistes_nest_3_sjh.eps} &
      \includegraphics[width=2.5cm, height=1.5cm]{/media/Preload/Pub2010/ThoughtsLinedUp/photos/Polybiaoccidentalis1.eps}\\
      \hline      
      Modelled after \alert{\textit{Polistes}} wasps:
      \textit{``global sensing no peer-to-peer communication''} & 
      
      Modelled after \protect\newline \alert{\textit{Polybia}} wasps: \protect\newline
     \textit{ ``local sensing local communication''}\\	      
      \hline
      Global \alert{broadcast} messaging & Local \alert{peer-to-peer} messaging\\
      \hline
      Communicate \protect\newline \alert{synchronously} & Communicate when peer(s) come \alert{in close contact} (inside range $r_{comm}$)\\
  		  \hline
      \end{tabular}
\end {scriptsize}
\end{columns}  
\end{frame}
%--------------------------------------------------------
\begin{frame}[t]{A Taxonomy of MRTA Solutions }
\begin{columns}
\column{0.5\linewidth}
%\vspace*{-0.8cm}
\begin{figure}
\centering
%\subfigure[\small Changes in task-urgency (CCM)]{
\includegraphics[width=0.99\linewidth]
{/media/Preload/Pub2010/RAS-Draft/images/taxonomy-ta-comm-OK.eps}
%}
\caption{\scriptsize Classification of MRTA solutions based on task-allocation and communication strategies}
\end{figure}
%%
\column{0.5\linewidth}
\vspace*{-0.8cm}
\begin{figure}
\includegraphics[width=0.99\linewidth]{/media/Preload/Pub2010/RAS-Draft/images/taxonomy-comm-interaction-OK.eps}
\caption{\scriptsize Information flow caused by different levels of communication and interaction}
\end{figure}
\end{columns}
\end{frame}	
%%%===========================================================
\section{Implementation}
\begin{frame}[t]{Multi-robot control architecture}
\vspace*{-0.25cm}
\begin{tabular}{|c|}
\hline
\small Based on our \alert{Hybrid-event Driven Architecture on D-Bus}\footnote{\scriptsize Sarker \& Dahl. UKACC Int'l Conference on Control, CONTROL 2010.}\\
\hline 
\end{tabular} 
\begin{figure}
\centering
\includegraphics[width=0.75\textwidth, angle=0]
{/media/Preload/Pub2010/RAS-Draft/images/RIL-Expt-Setup1.eps}
%figure caption is below the figure
\caption{\scriptsize Hardware and software setup for centralized communication experiments} 
\label{fig:RIL-Expt-Setup1} % Give a unique label
\end{figure}
\vspace*{-0.5cm}
\hrule
\scriptsize $^1$Sarker \& Dahl. \textit{Proc. of UKACC Int'l Conference on Control, Coventry, UK} 2010.
\end{frame}
%---------------------------------------------------
%\begin{frame}[t]{Tracking e-puck robots}
%\begin{columns}
%\column{0.8\linewidth}
%\begin{figure}
%\centering
%\subfigure[SwisTrack multi-robot tracker]
%{
%\includegraphics[width=0.85\linewidth]
%{/media/Preload/Pub2010/RAS-Draft/images/SwisTrackScreenshot.eps}
%}
%\end{figure}
%\column{0.25\linewidth}
%\begin{figure}
%\subfigure[E-puck robot]{
%\includegraphics[width=2.8cm, height=2.5cm]{/media/Preload/Pub2010/RAS-Draft/images/epuck-happy.eps}
%}
%\label{fig:e-puck}
%\end{figure}
%%%
%\begin{figure}
%\subfigure[E-puck marker]{
%\includegraphics[width=2.5cm, height=1.8cm]{/media/Preload/Pub2010/RAS-Draft/images/20-31412.eps}
%}
%\label{fig:e-puck}
%\end{figure}
%\end{columns}
%\end{frame}
%%%===========================================================
\section{Results}
%---------------------------------------------------------------
\begin{frame}[t]{Results: Shop-floor Work-load and Active Workers }
\begin{columns}
\column{0.5\linewidth}
\begin{figure}
\centering
%\subfigure[\small Changes in task-urgency (CCM)]{
\includegraphics[width=0.85\linewidth]
{/media/Preload/Pub2010/RAS-Draft/images/SB-PlotUrgencyLog-2010May10-115549.eps}
%}
\caption{\scriptsize Changes in task-urgency}
\end{figure}
%%
\vspace*{-0.5cm}
\begin{scriptsize}
\alert{Shop-floor work-load:}\\
Sum of changes in task-urgencies of\\ all $M$ tasks at $(q+1)^{th}$ step:
\begin{equation} 
\Delta \Phi_{j, q+1} = \sum_{j=1}^{M} (\phi_{j, q+1} - \phi_{j, q})
\label{eqn:Delta-Phi}
\end{equation}
\end{scriptsize}
%%
\column{0.5\linewidth}
\vspace*{-1.2cm}
\begin{figure}
%\subfigure[\small Shop-floor work-load]{
\includegraphics[width=0.85\linewidth]{/media/Preload/Pub2010/RAS-Draft/images/SB-TaskUrgencyStat.eps}
\caption{\scriptsize Shop-floor work-load}
\label{fig:r2}
\end{figure}
\begin{scriptsize}
\vspace*{-0.45cm}
\alert{Active worker ratio:}\\
\begin{equation} 
\frac{\textit{Active workers in all tasks}}{\textit{Total available workers}}
\label{eqn:Delta-Phi}
\end{equation}
\end{scriptsize}
%%
\end{columns}
\end{frame}
%--------------------------------------------------------------
\begin{frame}[t]{Results: Shop-floor Work-load  and Active Workers Ratio in 4 tasks experiments with 16 robots}
\begin{columns}
\column{0.5\linewidth}
\vspace*{-0.8cm}
\begin{figure}
\centering
%\subfigure[\small Changes in task-urgency (CCM)]{
\includegraphics[width=0.7\linewidth]
{/media/Preload/Pub2010/RAS-Draft/images/SB-TaskUrgencyStat.eps}
%}
\caption{\scriptsize Shop-floor work-load under  centralized comms.}
\end{figure}
%%
\vspace*{-1cm}
\begin{figure}
\centering
%\subfigure[\small Changes in task-urgency (CCM)]{
\includegraphics[width=0.7\linewidth]
{/media/Preload/Pub2010/RAS-Draft/images/SB-WorkerRatio.eps}
%}
\caption{\scriptsize Active worker ratio under  centralized comms.}
\end{figure}
%%
\column{0.5\linewidth}
\vspace*{-0.8cm}
\begin{figure}
\includegraphics[width=0.7\linewidth]{/media/Preload/Pub2010/RAS-Draft/images/SD-TaskUrgencyStat.eps}
\caption{\scriptsize Shop-floor work-load under local comms.}
\end{figure}
\vspace*{-1cm}
\begin{figure}
\includegraphics[width=0.7\linewidth]{/media/Preload/Pub2010/RAS-Draft/images/SD-Local1m-Plasticity.eps}
\caption{\scriptsize Active worker ratio under local comms.}
\end{figure}
\end{columns}
\end{frame}
%---------------------------------------------------------------
\begin{frame}[t]{Results: Task-Performance}
\begin{table}
\begin{scriptsize}
\begin{center}
\caption{Shop-floor production and maintenance task performance}
\begin{tabular}{|m{1.2in}|p{0.6in}|p{0.6in}|p{0.6in}|p{0.6in}|}
\hline Experiment Series & \textit{Production \protect\newline delay (SD) s} & \textit{p-value} \protect\newline 1-tailed t-test\protect\newline (confidence) & \textit{Pending \protect\newline maintenance time (SD) s} & \textit{p-value} 1-tailed t-test\\ 
\hline \alert{8 robots, 2 tasks, \protect\newline centralized, sample n=5 } & 
555 (50) & 0.0 & 5 (5) & 0.0\\ 
\hline \alert{16 robots, 4 tasks, \protect\newline centralized, sample n=5  } & 825 (360) & \alert{0.2 (60\%)} & 15 (65) & 0.0 \\
\hline \alert{16 robots, 4 tasks, \protect\newline local  with range=0.5m, sample n=3} & 605 (180) & N/A & 25 (85) & N/A\\
\hline \alert{16 robots, 4 tasks, \protect\newline local with range=1m, sample n=3}  & 615 (200) & 0.0 & 10 (35) & 0.0\\
\hline
\end{tabular}
\label{table:motion-cmp} 
\end{center}
\end{scriptsize}
\end{table}
\end{frame}
%---------------------------------------------------------------
\begin{frame}[t]{Results: Task-specialization}
\begin{small}
\begin{columns}
\column{0.5\textwidth}
\alert{Overall group task-specialization} in terms of peak values of sensitization of all robots:
\vspace*{-0.25cm}
\begin{equation}
K^G_{avg} = \frac{1}{N}\sum_{i=1}^{N} \max_{j=1}^M\left ( k^i_{j, q} \right ) 
\label{eqn:K-G}
\end{equation}
\column{0.5\textwidth}
\alert{Time spent to reach peak sensitization values}\\ for all robots:
%%
\vspace*{-0.25cm}
\begin{equation}
Q^G_{avg}= \frac{1}{N}\sum_{i=1}^{N} q^i_{k=k_{max}}
\label{eqn:Q-G}
\end{equation}
\end{columns}
\end{small}
%%
\vspace*{-0.25cm}
\begin{table}
\begin{scriptsize}
\begin{center}
\caption{\scriptsize Task-specialization values of the robots}
\begin{tabular}{|m{1.2in}|c|m{0.7in}|c|m{0.7in}|}
\hline Experiment Series & $ K^G_{avg}$ (SD) & \textit{ p-value} \protect\newline 1-tailed t-test (confidence)  & $ Q^G_{avg}$ (SD) & \textit{ p-value} \protect\newline 1-tailed t-test \protect\newline (confidence) \\ 
\hline \alert{8 robots, 2 tasks, centralized, n=5} & 0.40 (0.08)& 0.0 & 38 (13) & \alert{ 0.001 (99.8\%)}\\ 
\hline \alert{16 robots, 4 tasks, centralized, n=5} & 0.30 (0.03) & \alert{0.2 (60\%)} &  18 (5) & \alert{0.2 (60\%)}\\
\hline \alert{16 robots, 4 tasks, local  with range=0.5m, n=3}  & 0.39 (0.17) & N/A & 13 (7) & N/A \\
\hline \alert{16 robots, 4 tasks, local  with range=1m, n=3}  & 0.27 (0.1)& 0.0 & 11 (5) & 0.0\\
\hline
\end{tabular}
\label{table:motion-cmp} 
\end{center}
\end{scriptsize}
\end{table}
\end{frame}
%---------------------------------------------------------------
\begin{frame}[t]{Results: Energy-usage}
\begin{table}
\begin{small}
\begin{center}
\caption{Sum of translations of robots in our experiments.}
\begin{tabular}{|p{1.5in}|m{1in}|m{0.9in}|}
\hline Experiment Series & Average \protect\newline translation \protect\newline(SD) m & \textit{p-value} \protect\newline 1-tailed t-test (confidence)\\ 
\hline \alert{8 robots, 2 tasks, centralized, n=5} & 2.631 (0.804) & N/A\\ 
\hline \alert{16 robots, 2 tasks, centralized, n=5} & 13.882 (3.099) & \alert{0.001 (99.8\%)}\\
\hline \alert{16 robots, 4 tasks, local  with range=0.5m, n=3} & 4.907 (1.678) & N/A\\
\hline \alert{16 robots, 4 tasks, local  with range=1m, n=3}  & 4.854  (1.592) & 0.0\\
\hline
\end{tabular}
\label{table:motion-cmp} 
\end{center}
\end{small}
\end{table}
\end{frame}
%---------------------------------------------------------------
\begin{frame}[t]{Results: Communication Loads in terms of Frequency of \texttt{TaskInfo} signalling}
\begin{columns}
\column{0.5\linewidth}
\vspace*{-0.8cm}
\begin{figure}
\centering
%\subfigure[\small Changes in task-urgency (CCM)]{
\includegraphics[width=0.7\linewidth]
{/media/Preload/Pub2010/RAS-Draft/images/SA-8Robot-SignalingFreqStat.eps}
%}
\caption{\scriptsize  Under 8 robots, centralized communication}
\end{figure}
%%
\vspace*{-1cm}
\begin{figure}
\centering
%\subfigure[\small Changes in task-urgency (CCM)]{
\includegraphics[width=0.7\linewidth]
{/media/Preload/Pub2010/RAS-Draft/images/SB-SignalingFreqStat.eps}
%}
\caption{\scriptsize Under 16 robots, centralized communication}
\end{figure}
%%
\column{0.5\linewidth}
\vspace*{-0.8cm}
\begin{figure}
\includegraphics[width=0.7\linewidth]{/media/Preload/Pub2010/RAS-Draft/images/SC-Local-500cm-SignalingFreqStat.eps}
\caption{\scriptsize Under  16 robots, local communication, range=0.5m}
\end{figure}
\vspace*{-1cm}
\begin{figure}
\includegraphics[width=0.7\linewidth]{/media/Preload/Pub2010/RAS-Draft/images/SD-Local-1m-SignalingFreqStat.eps}
\caption{\scriptsize Under  16 robots, local communication range=1m}
\end{figure}
\end{columns}
\end{frame}

%%%===========================================================
\section{Conclusions}
%%-----------------------------------
\begin{frame}[t]{Conclusions \& Future works}
\begin{block}{Conclusions}
\begin{itemize}
    \item \small \alert{AFM solves the MRTA issue} for a relatively large group.\\
%    $\rightarrow$ \small under both centralized and local communication strategies.
    \item \small \alert{Task-performance varies under different communication strategies}\\
    $\rightarrow$ \small for a reasonably large group, local  communication achieves similar task-performance and task-specialization comparing with a centralized counterpart, but \textit{significantly} reduces motions.    
    \item \small AFM can model complex \alert{multi-tasking environment}\\
    \item \small \alert{Maximizing information flow} may not be  useful\\
    %$\rightarrow$ \small  under a stochastic task-allocation process, more information tends to cause more task-switching behaviours. % that lowers the level of task-specialization.     
 \end{itemize}
 \end{block}
\begin{block}{Future works} 
 \begin{itemize}
    \item \small Deploy our task-allocation model \alert{in various task settings}\\
%    $\rightarrow$ \small e.g. dynamic tasks, co-operative tasks, heterogeneous tasks.
    \item \small Relate \alert{communication range} as a property of self-regulation\\
%    $\rightarrow$ \small as a property of self-regulation of individuals.
    \item \small \alert{Real-world implementation:} e.g. warehouse automation\\
    %$\rightarrow$ \small e.g. warehouse automation, manufacturing shop-floor or any other multi-tasking environment.
     %\item \small Study the \alert{role of formal structure} on informal self-organization.
    \end{itemize}
\end{block}
\end{frame}
%%%===========================================================
%\section{Contributions}
\begin{frame}[t]{General Contributions}
\begin{itemize}
    \item \normalsize \alert{Self-organization}\\ 
$\rightarrow$ \small Self-organized allocation may produce specialized workers even when the group size is \textit{small} ($<$ 10), unlike assuming generalist workers prevents specialization in small groups\footnote{\scriptsize Garnier \textit{et al. Swarm Intelligence} 1:1, 2007.}.
    \item \normalsize \alert{Task-allocation}\\
$\rightarrow$ \small  Local communication in task-allocation may outperform centralized one in terms of energy usage.
\item \normalsize \alert{System development}\\
$\rightarrow$ \small Bottom-up de-coupled construction of \textit{large} artificial system yields higher advantages particularly, flexibility and integration with inter-operable elements.%in achieving self-regulated MRTA.
\end{itemize}
%
\hrule
\scriptsize $^2$Garnier \textit{et al. Swarm Intelligence} 1:1, 2007.
\end{frame}	
%----------------
\begin{frame}[t]{Specific Contributions}
\begin{itemize}
    \item \alert{Interpreted AFM}\\ 
$\rightarrow$ \small as a basic mechanism for multi-robot task-allocation%, specifically for abstract task-allocation in mult-tasking environment.
    \item \normalsize \alert{Validated the effectiveness of AFM}\\
$\rightarrow$ \small with reasonably \textit{large} number of real robots % i.e., 16 e-puck robots.
\item \normalsize \alert{Compared centralized \& local communication in MRTA}\\%in achieving self-regulated MRTA.
$\rightarrow$ \small modelled after social wasps: \textit{Polistes} and \textit{Polybia}
\item \normalsize \alert{Developed a \textit{flexible} multi-robot control architecture}\\ 
$\rightarrow$ \small using \textit{D-Bus} inter-process communication % technology.
\item \normalsize \alert{Classified MRTA solutions focusing three major issues:}\\ 
\begin{enumerate}
\begin{scriptsize}
\item Organization of task-allocation
\item Communication and
\item Interaction
\end{scriptsize}
\end{enumerate}
\end{itemize}
\end{frame}
%%%===========================================================
%%\section{Future works}
%\begin{frame}[t]{Future works}
%  \begin{itemize}
%    \item \normalsize \alert{Deploying our task-allocation model in various task settings}\\
%    $\rightarrow$ \small e.g. dynamic tasks, co-operative tasks, heterogeneous tasks.
%    \item \normalsize \alert{Find optimum communication range}\\
%    $\rightarrow$ \small as a property of self-regulation of individuals.
%    \item \normalsize \alert{Real-world implementation}\\
%    $\rightarrow$ \small e.g. warehouse automation, manufacturing shop-floor or any other multi-tasking environment.
%     \item \normalsize \alert{Studying the role of formal structure on non-formal self-organization}\\
%    $\rightarrow$ \small see in next few slides.
%    \end{itemize}
%\end{frame}

%%%%%%%%%%%%%%
\end{document}


